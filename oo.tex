\let\negmedspace\undefined
\let\negthickspace\undefined
\documentclass[journal,12pt,twocolumn]{IEEEtran}
\usepackage{cite}
\usepackage{amsmath,amssymb,amsfonts,amsthm}
\usepackage{algorithmic}
\usepackage{graphicx}
\usepackage{textcomp}
\usepackage{xcolor}
\usepackage{txfonts}
\usepackage{listings}
\usepackage{enumitem}
\usepackage{mathtools}
\usepackage{gensymb}
\usepackage{comment}
\usepackage[breaklinks=true]{hyperref}
\usepackage{tkz-euclide} 
\usepackage{listings}
\usepackage{gvv}                                        
\def\inputGnumericTable{}                                 
\usepackage[latin1]{inputenc}                                
\usepackage{color}                                            
\usepackage{array}                                            
\usepackage{longtable}                                       
\usepackage{calc}                                             
\usepackage{multirow}                                         
\usepackage{hhline}                                           
\usepackage{ifthen}                                           
\usepackage{lscape}
\setlength{\arrayrulewidth}{0.5mm}
\setlength{\tabcolsep}{18pt}
\renewcommand{\arraystretch}{1.5}

\newtheorem{theorem}{Theorem}[section]
\newtheorem{problem}{Problem}
\newtheorem{proposition}{Proposition}[section]
\newtheorem{lemma}{Lemma}[section]
\newtheorem{corollary}[theorem]{Corollary}
\newtheorem{example}{Example}[section]
\newtheorem{definition}[problem]{Definition}
\newcommand{\BEQA}{\begin{eqnarray}}
\newcommand{\EEQA}{\end{eqnarray}}
\newcommand{\define}{\stackrel{\triangle}{=}}
\theoremstyle{remark}
\newtheorem{rem}{Remark}
\begin{document}


\title{Waves(20) 11.15}
\author{EE23BTECH11051-Rajnil Malviya}
\date{January 2024}



\maketitle

\subsection*{\textit{Question :-}}
A train, standing at the outer signal of a railway station blows a whistle of frequency
400 Hz in still air. (i) What is the frequency of the whistle for a platform observer
when the train (a) approaches the platform with a speed of $10 ms^{-1} $, (b) recedes
from the platform with a speed of $10 ms^{-1} $? (ii) What is the speed of sound in each
case ? The speed of sound in still air can be taken as $340 ms^{-1} $.

\bigskip
 This problem requires knowledge of \textit{Doppler Effect} , So first we will learn Doppler effect and then we will solve our problem . Before learning Doppler effect , we will also understand Sound Waves .
 \begin{table}[h!]
   
        \begin{tabular}{ | m{1.0cm} | m{4cm} | } 
  \hline
 Symbol & Description \\ 
 \hline
 $y(t)$ & instantaneous displacement of wave \\
 \hline
$f_n $& natural frequency \\
\hline
$f_r $& received frequency \\
\hline
$A$ & amplitude of wave \\ 
\hline
 $\phi $& phase difference of wave \\
\hline
$t $& time  \\
\hline
$\lambda $& wavelength of wave \\
\hline
$T $& time period of wave \\
\hline
\end{tabular}\\
\caption{}
\label{Table:1}
       
    \end{table}
\subsection*{Equation of Sound Wave :-}
Sound Wave is transmission of energy ; sound wave depends on many parameters . A general equation of sound wave is shown below 
\begin{equation} \label{eq1}
y(t) = A\sin{( 2 \pi ft + \phi )}
\end{equation}

 \begin{table}[h!]
   
        \begin{tabular}{ | m{1.0cm} | m{3cm} |m{1.0cm}| } 
  \hline
 Symbol & Description  & Value\\ 
 \hline
 $f$ & frequency of source & 400Hz\\
\hline
$v$& velocity of air   &  $340 ms^{-1} $\\
\hline
$v_o$& velocity of observer   &  $0 ms^{-1} $\\
\hline
$v$& velocity of source   &  $10 ms^{-1} $\\
\hline

\end{tabular}\\
\caption{}
\label{Table:2}
       
    \end{table}

From equation 1 , equation of sound wave when whistle is blown by
train is 
\newpage
$$y(t) = A\sin{( 2 \pi \times400\times t + \phi )} $$ 
\;\;\;\;\;\;\;\;\;\;\;\;\;\;\;\;\;\;\;\;for this case $f\;=\;400Hz$\\
\subsection*{\textit{Doppler Effect for Sound Waves :-}}
Doppler effect for sound wave refers to change in frequency or pitch of sound wave observed by an observer when there is a relative motion between observer and source .
\subsection*{\textit{General \;Equation \;of \;Doppler :-}}
Now we will see formulas for Doppler effect in different situations . \\
\begin{align}{\label{eq2}}f' = \frac{v'}{\lambda'}\end{align}
\begin{align}{\label{eq3}}f_n = \frac{v}{\lambda}\end{align}
\begin{align}{\label{eq4}}f' = \frac{v \pm v_o}{\lambda \pm v_s T}\end{align}
\begin{align}{\label{eq5}}f_n = \frac{1}{T}\end{align}
From equations 3 and 5 , then substituting in equation 4
\begin{align}{\label{eq6}}f' = \frac{(v \pm v_o)(f_n)}{v \pm v_s }\end{align}
Above equation is general equation for any case in doppler effect .
Signs of equation 6 depends on velocities of both observer and source .
\subsection*{\textit{Effect \;of  \;observer's  \;velocity\;on \;frequency :-}}
If observer is moving towards source and source is stationary ($v_s=0)$, then sound will reach observer faster , so observed frequency will increase. It means \begin{align}{\label{eq7}}f'>f_n\end{align} From equation 6 \begin{align}{\label{eq8}} f' = \frac{(v \pm v_o)f_n}{v}\end{align}\\
From equations 3 and 8 , 
\begin{align} \frac{(v \pm v_o)f_n}{v} > f_n\end{align}
\begin{align}{\label{eq10}} v \pm v_o > v\end{align}
 There must be (+) sign to satisfy equation 10 .
And for vice-versa case(observer is moving away) , so frequency will be less than $f_n$
So ,\begin{align}{\label{eq11}}f'<f_n\end{align}
From equation 8 ,
\begin{align} \frac{(v \pm v_o)f_n}{v} < f_n\end{align}
\begin{align}{\label{eq13}} v \pm v_o < v\end{align}
There must be (-) sign to satisfy equation 13 .
\subsection*{\textit{Effect \;of  \;source's  \;velocity\;on \;frequency :-}}
If source is moving towards stationary observer($v_o =0$), so $\lambda'$ in equation 2 will compress and denominator will decrease , so $f'$ will increase , \begin{align}f'>f_n\end{align}From equation  6
\begin{align}f' = \frac{(v)f_n}{v \pm v_s}\end{align}\\
From equations  8 and 3
\begin{align} \frac{v f_n}{v \pm v_s} > f_n\end{align}
\begin{align}{\label{eq17}} \frac{v}{v \pm v_s} > 1\end{align}
 There must be (-) sign to satisfy equation  17 .
And for vice-versa case(source is moving away) , so wavelength will increase and denominator in equation 4 increases so $f'$ will decrease \\
\begin{align}f'<f_n\end{align}
From equation 6
\begin{align}{\label{eq19}} f' = \frac{(v)f_n}{v \pm v_s}\end{align}\\
From equations 19 and 3 , 
\begin{align} \frac{v f_n}{v \pm v_s} < f_n\end{align}
\begin{align}{\label{eq21}} \frac{v}{v \pm v_s} < 1\end{align}
 There must be (+) sign to satisfy equation 21.\\\\
 $v_o$ and $v_s$ directly affect equation 6 , independent of each other . We can change signs of numerator and denominator independently by analysing situation .\\\\
A table is given below , which includes all cases with different situations .
    \begin{table}[h]
   
        
\setlength{\arrayrulewidth}{0.5mm}
\setlength{\tabcolsep}{14pt}
\renewcommand{\arraystretch}{0.8}


\begin{tabular}{ |p{1cm}|p{1cm}|p{1.5cm}|p{1.5cm}|p{1}}
    \hline
    \multicolumn{4}{|c|}{frequencies observed in Different cases} \\
    \hline
    Doppler Shift &Stationary Observer &Observer moving towards Source &Observer moving away from Source\\
    \hline
    Stationary Source & $$f' = f_n$$& $$f' = \frac{(v+v_o) f_n}{v}$$&$$f' = \frac{(v-v_o) f_n}{v}$$\\
    \hline
    Source moving towards Observer &$$f' = \frac{v f_n}{v-v_s }$$&$$f' = \frac{(v+v_o) f_n}{v- v_s }$$&$$f' = \frac{(v-v_o) f_n}{v- v_S }$$\\
    \hline
    Source moving away from Observer&$$f' = \frac{v f_n}{v+ v_s }$$&$$f' = \frac{(v+v_o) f_n}{v+ v_s }$$&$$f' = \frac{(v-v_o) f_n}{v+ v_s }$$\\
    \hline
    \end{tabular}\\
\caption{}
\label{Table:3}
       
    \end{table}
    \newpage
Let's get back to our problem solution\\
(i).a When the train approaches the platform (i.e., the observer at rest),\\
\textit{Solution  :-}\\
So here source is approaching , wavelength will decrease and frequency will increase , so we have to increase $f' $\\
From Table:3
\begin{align}{\label{eq22}}f' = \frac{v f_n}{v- v_s }\end{align}
On Substituting in equation 22
$$f'_a=400(\frac{340}{340-10})$$
$$f'=412.1212$$
\bigskip
b. When the train recedes the platform (i.e., the observer at rest), \\
\textit{Solution  :-}\\
It is vice-versa of above, From Table:3
\begin{align}f'=f_n(\frac{v}{v+v_s})\end{align}
$$f'=400(\frac{340}{340+10})$$
$$f'=388.5714$$\\
(ii) The speed of sound in each will be same.It is $340  ms^{-1}$ in each case.\\\\
 \begin{table}[h!]
   
        \begin{tabular}{ | m{2.5cm} | m{2.5cm} | } 
  \hline
 Transmitted Signal & Received Signal\\
 \hline
 Source transmits & Source will receive  \\
  a signal with & reflection of its \\
   frequency &transmitted Signal\\
   \hline
    & \\
$A\sin{(2 \pi f t +\phi)}$&$A\sin{(2 \pi f_r t +\phi)} $ \\
    & \\
\hline
\end{tabular}\\
\caption{}
\label{Table:4}
       
    \end{table}
From equation 6
we will interchange $v_o\; and\; v_s$ 
\begin{align}{\label{eq24}}f_r = \frac{(v \pm v_s)(f')}{v \pm v_o }\end{align}
For signs , we will use same logic, if someone is approaching another one , so definitely it will increase frequency($f_r$) and if receding so it will decrease frequency($f_r$) . \\\\
 In our problem  if , train approaches with $v_s10 ms^{-1} $ and if platform is taken as obstacle with $v_o=0$, \\
 Received frequency by platform is \textit{f'} 412.1212   , \\
 Here train(source) is approaching , so $v_s$ will try to increase $f_r$\\
On substituting in  equation 24
 $$f_r = \frac{(v \pm v_s)(f')}{v }$$
 To increase $f_r$ , there must be (+) sign ;
 $$f_r = \frac{(340 + 10)(412.1212)}{340 }$$
 $f_r$=424.2424\\ \\So Received signal will be , \\
 From table Table:4
 $$y_r(t) = A\sin{( 2 \pi  (424.2424)t + \phi )}$$
 if train is receding with $10 ms^{-1} $ , so $v_s$ will try to decrease $f_r$
 so we will use (-) sign in equation 24 with $v_o=0$\\
  $$f_r = \frac{(v -v_s)(f')}{v }$$
  $f'$=388.5714 \\
  on substituting in equation  24
  $$f_r = \frac{(340 -10)(388.5714)}{340 }$$
  $f_r=377.1428$
 \\ So received signal will be , \\
 From table Table:4
 $$y_r(t) = A\sin{( 2 \pi  (377.1428)t + \phi )}$$

\end{document}
